\documentclass{6042}

\author{Ian Miller}
\problemset{3}


\begin{document}


\problem{1}{Andy Kaspers}


\problempart{a}
$\sum{F_i^2} = F_i * F_{i+1}$

Proof by induction

Let P represent the following proposition: $\sum{F_i^2} = F_i * F_{i+1}$

In this proof by induction, I will prove through the use of induction that P is indeed true.

\textbf{Base case:} i = 1

$\sum{F_1^2} = 1$

$F_1 * F_2 = 1$

The base case holds.

\textbf{Inductive Step:} Assume that $\sum{F_i^2} = F_i * F_{i+1}$ for all $i > 0$.

I will now show that P implies P+1.

We know that $\sum{F_i^2} = F(1)^2 + .... + F(i)^2 = F(n) * F(n+1)$

Therefore, the n+1 case looks like the following:

$\sum{F_i^2} = F(1)^2 + .... + F(i)^2 + F(i+1)^2 = F(n) * F(n+1) + F(n+1)^2$

$= F(n) * F(n+1) + (F(n+1) * (F(n+1)))$

$= F(n+1) * F(n+2)$

This proves our inductive hypothesis and proves that P implies P+1. Therefore P holds. QED


\problempart{b}

Let $S = \sum{F_i/2^i}$ 

$S = F_0 + 1/2*F_1 + \sum_{i=2}^\infty * (F_{i-1} + F_{i-2})(1/2)^i$

$S = 1/2 + \sum_{i=2}^\infty F_{i-1}(1/2)^i + \sum_{i=2}^\infty F_{i-2}(1/2)^i$

$S = 1/2 + 1/2*S + 1/4 * S$

$S -1/2*S - 1/4*S = 1/2$

$S * (1 - 1/2 - 1/4) = 1/2$

$1/4 * S = 1/2$

$S = 2$



$\sum_{n=0}^\infty F_n/k^n = k / (k^2 - k - 1) = 2 / (4 - 2 - 1) = 2$

\problem{2}{None}


\problempart{a}

   \begin{tabular}{ |c|c|c|c|c|c| } 
    \hline
     & $\Theta(x^2)$ & $O(2^x)$ & $\omega(log_2 x)$ & $\Omega(e^{\sqrt{x}})$ & $o(x)$\\ \hline
    ln(x) & NO & YES & YES & NO & YES\\ \hline
    $1/2 (x-1)^2$ & NO & YES & NO & NO & NO\\\hline
    9 & NO & YES & NO & NO & NO\\ \hline
    $2^{2x}$ & NO & NO & NO & YES & NO\\ \hline
    $\sqrt{5x + 3}$ & NO & YES & NO & NO & YES\\ \hline
    $\alpha(x)$ & NO & NO & NO & YES & NO\\
    \hline
   \end{tabular}
\problempart{b}

Let P represent the following proposition: $f(n) = O(g(n))$

Let Q represent the following proposition: $(f(n))^2 = O((g(n))^2)$

In this problem, I will prove that P $->$ Q.

Based on the definition provided in the problem, we can rewrite P as:

$\exists{c} > 0. \exists{n_0} \in\natural.\forall{n} > n_0. |f(n)| <= c * g(n)$

If you square both sides of the inequality, you get $f(n)^2 <= c^2 * g(n)^2$

which is equivalent to Q. Therefore P implies Q. QED


\problem{3}{Textbook, pg 253}


\problempart{a}

In this problem, I will prove an estimate of the following sum: $\sum_{i=1}^\infty 1 / (2*i - 1)^3$

Let $S = \sum_{i=1}^\infty 1 / (2*i - 1)^3$

and $I = \int_1^\infty 1 / (2*i - 1)^3$

In this case, since the function is not increasing, $I + 1 / (2*i - 1)^3 <= S <= I + 1 / (2*i - 1)^3$

$I = \int_1^\infty 1 / (2*i - 1)^3$

$= -1 / (4*(2x-1)^2)$ (from 1 to infinity)

$= 1/4$

QED


\problem{4}{Savannah Tynan, Textbook pg 292}

\problempart{a}

Proof by Worst Case Scenario

Let P represent the following proposition: \textit{Show that 2n - 3 comparisons are enough to merge three sorted lists each containing n/3 items.}

In this problem, I will prove P by calculating the number of comparisons needed to sort in the \textbf{worst-case scenario}.

If I can show that the number of calculations in this worst-case scenario is less than 2n-3, we have proven that there is never a situation where P is false, which means P must be true.

In the worst case:

A) it will take n/3 comparisons to sort each of the three sublists, recursively. That is a total of n comparisons.

B) it will take n - 3 comparisons to sort (n items are emitted in total, and once a sublist becomes empty, it takes at most 3 comparisons to sort those 3 numbers)

The reason it takes at most 3 comparisons to compare 3 numbers (for example, the last element in each of the three sublists) can be described in the following way:

Take x, y, and z as 3 integers that each belong to their own sublists. x must be compared with y and z. That takes 2 comparisons. Then, y and z must be compared. That makes 3 comparisons in total.

If A) and B) are added together, that gets us to n + n -3 comparisons, or $2n -3$ comparisons, proving P. QED.


\problempart{b}

$T_1 = 0$

$T_2 = 3T_1 + 2 - 3 = -1$

$T_3 = 3T_2 + 4 - 3 = -2$

$T_4 = 3T_2 + 8 - 3 = -1$

$T_5 = 3T_2 + 16 - 3 = -10$

$T(n) = 3T(n/3) + 2n -3$

$T(n) = 3T(n/3) + 2n -3$

$ = 3(3T(n/9) + 2n/3 - 3) + (n - 3)$

$ = 9T(n/9) + 2n - 9) + (n - 3)$

$ = 9(3T(n/27) + 2n/27 - 3) + (n-9) + (n - 3)$

$ = 27T(n/27) + 2n - 27) + (n - 27) + (n - 9) + (n - 3)$

$T_n = 3^kT_n/{3^k} + kn - 3^k + 1$

This is not an asymptotic improvement over MergeSort.

\end{document}

