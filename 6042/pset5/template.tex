\documentclass{6042}

\author{Ian Miller}
\problemset{5}


\begin{document}


\problem{1}{None}


\problempart{a}
Fuller House = (4 cards of one rank) + (2 cards of another rank)

Let $P =$ the chance of getting a Fuller House if you randomly flipped over 6 cards

Let $T =$ the total number of 6-card hands ($52*51*50*49*48*47/6!$)

The number of possible Fuller House hands is equivalent to:

$P * T$

We already have T so in order to solve the problem, so we only need to calculate P.

$P = (1 * (3/51) * (2/50) * (1/49)) * ((44/48) * (3/47))$



\problempart{b}

I'm going to use the same approach as in part A.

Three Pair = (2 cards of one rank) + (2 cards of another rank) + (2 cards of another rank)

Let $T =$ the total number of 6-card hands ($52*51*50*49*48*47/6!$)

The number of possible Three Pair hands is equivalent to:

$P * T$

We already have T so in order to solve the problem, so we only need to calculate P.

$P = (1 * (3/51)) * ((48/50) * (3/49)) * ((44/48) * (3/47))$

\problempart{c}

It's a little harder to calculate the probability the same way as in part A and B.

There are 2 variables I'm worried about to find the total number of Imperial Flushes:

1- High card

2- Suit 

For each high card in an Imperial Flush, there are 3 others possible because suit can be changed.

Since there are 9 possible high cards (6, 7, 8, 9, 10, J, Q, K, A), there are $4 * 9 = 36$ possible Imperial Flushes.

\problempart{d}

I'm going to use the same approach as in part A and B.

Basically-a-Flush = AT LEAST 5 cards in the hand have the same suit

Let $T =$ the total number of 6-card hands ($52*51*50*49*48*47/6!$)

The number of possible Basically-a-Flush hands is equivalent to:

$P * T$

We already have T so in order to solve the problem, so we only need to calculate P.

$P = (1 * (12/51) * (11/50) * (10/49)) * (9/48) * 1)$

$P = (12/51) * (11/50) * (10/49)) * (9/48)$

\problem{2}{None}


The pigeonholes are the following: for each odd number x in the set ${1, 2, 3,...., 2n}$, we
can make the set $S_x = {x, 2x, 4x, 2^y(x),...}$

This gives us n pigeonholes. The problem dictates that we have to pick n+1 numbers.
Any pair of numbers that satisfies the problem constraint (quotient is a power of two) will be in the same set $S_x$.

Therefore, any pair of numbers will contain one number that divides the other. Each number that we pick is a pigeon. And because of 
each number will be in the same $S_x$ as another number that divides it, its quotient will be a power of two. 

Therefore, we have n+1 pigeons and n pigeonholes.

\problem{3}{Sophia Chan, Julian Hamelberg}


\problempart{a}

First, we must construct a bijection to represent donut selection. This bijection will map the set of all possible
ways to select n donuts from k possible flavors to the set of distinct n+k-1 bit sequences with n zeroes and k-1 ones.

There are n+k-1 bits regardless of how the zeroes are arranged, assuming there are k-1 ones. That sequence is an element of the set of all
distinct n+k-1 bit sequences with n zeroes and k-1 ones.

This is a bijection because every possible arrangement of ones and zeroes is mapped to from one of the elements in the set of all possible ways to select
n donuts from k flavors. Therefore, we can say that there are 
(
\(
  {\begin{array}{c}
   n+k-1\\
   k-1\\
  \end{array} }
\)
)

ways to select n donuts from k possible flavors.

\problempart{b}

First, we must construct a bijection to represent donut selection. This bijection will map the set of all possible
ways to select 12 donuts from 5 possible flavors to the set of distinct 11 bit sequences with 7 zeroes and 4 ones.

If we are required to select 1 donut in each flavor, that means that 5 of the zeroes are already accounted for. This means that
you only have flexibility with those last 7 donuts. The bit length is now 11. In that 11 bit sequence, there will 7 zeroes and 4 ones
meaning that there are (
   \(
     {\begin{array}{c}
      11\\
      4\\
     \end{array} }
   \)
   )
ways to select 12 donuts from 5 flavors, based on our work from part A. 

\problempart{c}

In order for there to be either 0 or an even number of donuts per flavor in our selection, we must pick donuts 2 at a time.

This is equivalent to making the selection size 6 (half of 12). Therefore according to the stars and bars formula in part A,
there are \(
   {\begin{array}{c}
    10\\
    4\\
   \end{array} }
 \)
 )
ways to make a selection that meet the problem constraints. 

\problempart{d}

The number of ways to select 12 donuts from 5 flavors requiring 3 distinct flavors is equivalent to the total 
number of ways to select 12 donuts from 5 flavors minus the number of ways to select 12 donuts from 5 flavors with at most 2 flavors.

The total number of ways to select 12 donuts from 5 flavors with no constraint is the following:

(
\(
  {\begin{array}{c}
   12+5-1\\
   5-1\\
  \end{array} }
\)
)

The total number of ways to select 12 donuts from 5 flavors with at most 2 flavors is the following:

(
\(
  {\begin{array}{c}
   12+2-1\\
   2-1\\
  \end{array} }
\)
)
*
(
\(
  {\begin{array}{c}
   5\\
   2\\
  \end{array} }
\)
)

Therefore, the total number of ways to select 12 donuts from 5 flavors, if we required
at least 3 distinct flavors is 


(
\(
  {\begin{array}{c}
   12+5-1\\
   5-1\\
  \end{array} }
\)
)
-
(  
(
   \(
     {\begin{array}{c}
      12+2-1\\
      2-1\\
     \end{array} }
   \)
   )
   *
   (
   \(
     {\begin{array}{c}
      5\\
      2\\
     \end{array} }
   \)
   )
)
\end{document}
