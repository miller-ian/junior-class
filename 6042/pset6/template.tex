\documentclass{6042}

\author{Ian Miller}
\problemset{6}


\begin{document}


\problem{1}{None}


\problempart{a}

Ignoring vowels, there are $21!$ ways to arrange consonants.

There are 21 possible landing spots for a vowel (in between each consonant). And there are 5 vowels to assign.

So, there are $\frac{(21!)^2}{16!}$ possible ways to order the 26 letters of the alphabet so that no two vowels appear consecutively and the last letter is not a vowel.

\problempart{b}

We are going to attach 2 arbitrary consonants to the end of each vowel. So we have:

(a, C1, C2)

(e, C3, C4)

(i, C5, C6)

(o, C7, C8)

(u, C9, C10)

There are (21 choose 10) assignments for C1-C10. That's equivalent to 352,716.

Once we assign C1-C10, the problem becomes very similar to part A. We have 11 consonants remaining. There are 11! ways to arrange those consonants.

There are 11! ways to arrange the remaining consonants, relative to each other.

There are 12 possible landing spots for the 3-letter units. So, there are $\frac{352,716 * 11! * 12!}{5!}$ possible ways to order the
26 letters of the alphabet so that there are at least two consonants immediately following each vowel.

\problempart{c}

$\frac{(2n)!}{2^nn!}$

If you arrange all students in a single line, and take the first two off the front of the line,
you can generate a lineup. There are $(2n)!$ possible lineups. But some of those lineups will repeat.
So we need to divide by something in order to shave off those duplicates (since order doesn't matter).

As you go down that initial single line, there are $2^n$ ways you could order each pair (i.e. AB becomes BA while all the other pairs remain the same).

You can also rearrange the pairs relatve to each other (i.e. AB CD could begin CD AB). There are n! ways you could do that.

Thus, we get $\frac{(2n)!}{2^nn!}$ ways to pair 2n students.

\problempart{d}

This question is essentially asking us to find the total number of combinations (with replacement) of elements that make up a subset of a larger set of elements.

This is also called multichoose.

There are $\frac{(n + 10-1)!}{10!(n-1)!}$ types of n-digit sequences.


\problem{2}{Albert Garcia}

\problempart{a}
The pigeons are represented as the 82 rows. 

The pigeonholes are represented as the total number of colorings for each row

Each row has 4 boxes and, with 3 possible colors, there are 81 total colorings for that row.

Since there are 81 pigeonholes and 82 pigeons, at least 2 rows must have the same coloring.

\problempart{b}

We know that 2 rows must have the same coloring.

And since there are only 3 colors for 4 boxes, in every row, there must be a repeated color.

This means that for those 2 repeated rows, a rectangle can be made where the 4 corners of that rectangle are the same color.

\problempart{c}
In this situation, the pigeons represent the 19 rows.

The pigeonholes represent the total number of ways that rows can be arranged with at least 2 of the same colors. 

As mentioned in part b, at least 1 color must be repeated in a row. There are also 6 ways to pick two positions in a row of length 4 and color them the same.
That means that the total number of ways we can arrange rows with at least two of the same color is 18.

We have more pigeons than pigeonholes. Part B holds even which R only has 19 rows.

\problem{3}{Textbook, page 334}

\problempart{a}
In order to figure out how many 10-digit passwords don't contain forbidden sequences, we'll find the total number of 
10-digit passwords possible and subtract the number of passwords that do contain a forbidden sequence.

total number of 10-digit passwords = 10!

total number of passwords that contain "6042" = 7*6! = 7!

total number of passwords that contain "18062" = 6*5! = 6!

total number of passwords that contain "38476" = 6*5! = 6!

Total number of passwords that don't contain forbidden sequences = 10! - 7! - 6! - 6!

\problempart{b}
Like in part A, in order to figure out how many 10-digit passwords don't contain forbidden sequences, we'll find the total number of 
10-digit passwords possible and subtract the number of passwords that do contain a forbidden sequence.

total number of 10-digit passwords = $10^{10}$

total number of passwords that contain "6042" and not other 2 forbidden passwords = $(7 * (10^6)) - 12$

total number of passwords that contain "18062" and not other 1 forbidden password = $(6 * (10^5)) - 2$

total number of passwords that contain "38476" = $(6 * (10^5))$

Total number of passwords that don't contain forbidden sequences = $10^{10} - ((7 * (10^6)) - 12) - ((6 * (10^5)) - 2) - (6 * (10^5))$



\problem{4}{Textbook, page 342}

\problempart{a}
\textit{Proof.} We give a combinatorial proof. Let \textit{S} be the set of all k-element groups that satisfy the following 2 conditions:

1- each element can be selected from n choices, and
2- 1 element from within each k-element subset must be singled out as the "leader" of that subset

By counting, we determine that in this situation, $S = \sum_{k=0}^{n} k * (n$ choose $k)$

From another perspective, we can select the "leader" of a subset in $n$ ways. Following that pick, there are $2^(n-1)$ ways to select the remaning elements.
In this way, there are $n * 2^(2n-1)$ ways to get \textit{S}.

Equating these two expressions for $|S|$ proves that $\sum_{k=0}^{n} k * (n$ choose $k) = n * 2^(2n-1)$
$n * 2^(n-1)$ since there are $2^(n-1)$ ways to choose k elements out of a set of n elements. 

QED
\problempart{b}
\textit{Proof.} We give a combinatorial proof. Let \textit{S} be the set of all n-element groups that can be created from a set of 2n elements.
We can create $(2n$ choose $n)$ subsets so in this way, $S = (2n$ choose $n)$.

From another perspective, $(n $choose $k) = (n$ choose $(n-k))$. 

So, $\sum_{k=0}^{n} (n $choose $k) (n$ choose $n-k) = \sum_{k=0}^{n} (n$ choose$ k) (n$ choose $k)$

which is equivalent to (2n choose n). 

Equating these two expressions proves that $(2n$ choose $n) = \sum_{k=0}^{n} (n $choose $k) (n$ choose $n-k)$

QED
\end{document}
