\documentclass{6042}

\author{Ian Miller}
\problemset{7}


\begin{document}


\problem{1}{None}


\problempart{a}
Independence means that $P(X|Y) = P(X)$ for all values of X and Y.

If $C = 3$, the following table shows the decimal probabilities for each event:
$\begin{bmatrix}
    H1 & 1.0 \\
    H2 & 1.0 \\
    H3 & 1.0 \\
    M & 1.0 \\
    S & 1.0
\end{bmatrix}$

If $C = 1$, the following table shows the decimal probabilities for each event:
$\begin{bmatrix}
    H1 & 0.33 \\
    H2 & 0.33 \\
    H3 & 0.33 \\
    M & 0.0 \\
    S & 1.0
\end{bmatrix}$

If $C = 0$, the following table shows the decimal probabilities for each event:
$\begin{bmatrix}
    H1 & 0.0 \\
    H2 & 0.0 \\
    H3 & 0.0 \\
    M & 1.0 \\
    S & 0.0
\end{bmatrix}$

For each variable, depending on the value of C, probabilities can change. Therefore, none of the variables are independent of C.


\problempart{b}

M and S will be pairwise independent is the following is True:

$Pr[M \cap S] = Pr[M] * Pr[S]$

$Pr[M \cap S] = 0.125$

$Pr[M] * Pr[S] = .25 * 0.5 = 0.125$

Since the independence formula holds True, M and S are pairwise independent.


\problempart{c}
$Pr[H1 \cap H2 \cap H3] = Pr[H1 \cap H2 \cap S] = Pr[H2 \cap H3 \cap S] = Pr[H1 \cap H3 \cap S] = 1/8$

Therefore, H1, H2, H3, and S are 3-wise independent.

However, 

In the case that H1, H2, and H3 all equal 1: Pr[C] = 1

In the case that H1, H2, and H3 all equal 0: Pr[C] = 0

Therefore, H1, H2, H3, and S are NOT mutually independent.

\problem{2}{Albert Garcia}

\problempart{a}
PDF$_{x+1}(a) -> \begin{Bmatrix}
    2 & 1/3 \\
    1 & 2/3
\end{Bmatrix}$

PDF$_{y}(a) -> \begin{Bmatrix}
    1 & 1/6 \\
    2 & 1/6 \\
    3 & 1/6 \\
    4 & 1/6 \\
    5 & 1/6 \\
    6 & 1/6
\end{Bmatrix}$

Because X and Y are independent, we can calculate the probability density function Z by multiplying
corresponding matrix elements.

PDF$_{Z}(a) -> \begin{bmatrix}
    1 & 1/9 \\
    2 & 1/6 \\
    3 & 1/9 \\
    4 & 1/6 \\
    5 & 1/9 \\
    6 & 1/6 \\
    8 & 1/18 \\
    10 & 1/18 \\
    12 & 1/18 \\
\end{bmatrix}$

\problempart{b}
CDF$_{D}(a) -> \begin{bmatrix}
    1 & Pr[D <= 1] ~ 0.6666 \\
    2 & Pr[D <= 2] ~ 0.8888 \\
    3 & Pr[D <= 3] ~ 0.9629 \\
    4 & Pr[D <= 4] ~ 0.9876 \\
    5 & Pr[D <= 5] ~ 0.9958 \\
    6 & Pr[D <= 6] ~ 0.9986 \\
    7 & Pr[D <= 7] ~ 0.9995 \\
    8 & Pr[D <= 8] ~ 0.9998 \\
    9 & Pr[D <= 9] ~ 0.9999 \\
    10 & Pr[D <= 10] = 1
\end{bmatrix}$

For all a's where a is in range of D, $\sum{PDF_{D}(a)} = 1$

Therefore, the probability of rolling a 10 is $1 - \sum_{1}^{9} PDF_{D}(a)$ approximately equals 0.0000508

\problem{3}{None}

\problempart{a}

\problempart{b}

\problempart{c}


\problem{4}{None}

\problempart{a}

\problempart{b}

\problempart{c}


\end{document}

